\documentclass{article}

\usepackage{formatting}

\begin{document}

\FirstPage{name of company}{address of company} {student number}

\newpage

\MemorandumHead{Supervisor's Name}

As we agreed, I have prepared the enclosed report, “\WTT,” for
my \term work report and for the TEAM. This
report, the NUMBER of four work reports that the Co-operative Education Program
requires that I successfully complete as part of my BMath Co-op degree
requirements, has not received academic credit.
\vskip 10pt
ROLE IN COMPANY, BRIEF DESCRIPTION OF DUTIES, PURPOSE OF REPORT must be added in a paragraph
\vskip 10pt 
The Faculty of Mathematics requests that you evaluate this report for command
of topic and technical content/analysis. Following your assessment, the report,
together with your evaluation, will be submitted to the Math Undergrad Office
for evaluation on campus by qualified work report markers. The combined
marks determine whether the report will receive credit and whether it will be
considered for an award.
Thank you for your assistance in preparing this report.

INSERT SIGNATURE HERE
\newpage

\header 
\tableofcontents
\listoffigures

\newpage
\section*{Executive Summary}
One of the most important components of the report is the Executive Summary. It should be written after the rest of the report has been written.

The Executive Summary should be complete in itself and may be consulted by readers to determine whether they need to read the whole report. It appears on a separate page.

Limit the Executive Summary to one page and briefly present

the purpose of the report
the key points of the analysis
the highlights of the conclusions
the highlights of the recommendations
\newpage
\pagenumbering{arabic}
\section{Introduction}
The Introduction establishes the purpose of the report and conveys the contents of the Analysis. You should provide the reader with the following information:

necessary background information
assumptions used
major points covered in the report
the situation or problem that is analyzed
the purpose of the work report and the methodology used

\section{Analysis} 
\subsection{A subsection}
A description of the steps in a process is not sufficient. The following list provides examples of acceptable analytical content:

a discussion of cause and effect
a discussion of advantages and disadvantages
a comparison of two or more systems or products
The following are examples of acceptable analyses:

Why does a problem exist?
How does the problem affect specific jobs in the workplace?
How does the new system or product solve a problem?
What aspects of the problem have been improved? How?
What problems does the system or product not solve? Why not?
How can the system or product be improved?

\section{Conclusions}
The Conclusions section should be brief and should contain no new information. Conclusions should not make direct reference to sources, figures, or tables.

Each conclusion should follow logically from the facts and arguments presented in the Analysis section.

\section{Recommendations}
This section is optional because recommendations are not appropriate for all reports.

Recommendations are essentially speculative but should be brief and should follow logically from the conclusions. Include comments derived from experience that may improve future activities of the company.

\newpage
\section*{References}
Whenever possible, support statements with concrete, specific examples. If you refer to a published work, cite the reference in the text even if the reference is not a direct quotation. Include the author and year of publication.

\newpage
\section*{Acknowledgments}

\end{document}
